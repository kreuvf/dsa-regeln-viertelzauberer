\section{Überarbeitete Viertelzaubererregeln}
Bevor ich in die Details gehe, möchte ich die Eckpfeiler dieser Regeln kurz erläutern und am Ende dieser Regeln mit einigen Beispielen die neuen Regeln in Aktion zeigen. Den Abschluss bilden ausführliche Kommentare zu diesen Regeln -- ähnliche Kommentare der Autoren würde ich mir auch für die offiziellen Regelwerke wünschen.

Der Vorteil \enquote{Viertelzauberer} wird auch weiterhin benötigt, wenn einer der davon abhängigen Vorteile \enquote{Meisterhandwerk}, \enquote{Schutzgeist} oder \enquote{übernatürliche Begabung} erworben werden soll. Mit übernatürlicher Begabung geht die neue Repräsentation \enquote{Magiedilettant (Name)} einher, die für jeden Viertelzauberer einmalig ist. Übernatürliche Begabungen stellen keine vollwertigen Zauber mehr dar, sondern stark eingeschränkte Zauberwirkungen, die auf den jeweiligen Viertelzauberer angepasst sind.

\subsection{Vorteile}
Der Vorteil \enquote{Viertelzauberer} kostet \SI{3}{\GP}, solange keine übernatürliche Begabung vorhanden ist und der Spieler möchte, dass der Charakter nichts von seiner magischen Begabung weiß. Das heißt, dass sowohl Meisterhandwerk als auch Schutzgeist ausschließlich intuitiv, also nach Meisterentscheid, eingesetzt werden. Ob sich ein unbewusster Viertelzauberer im Laufe seines Lebens seiner Kräfte bewusst werden kann und diese dann auch gezielt einsetzen kann, ist Meisterentscheid. In allen anderen Fällen kostet der Vorteil \enquote{Viertelzauberer} \SI{5}{\GP}. Die Auswirkungen des Vorteils \enquote{Viertelzauberer} lauten wie folgt:
\begin{itemize}
	\item Astralenergie: Grundvorrat − \SI{3}{\AsP}
	\item Magieresistenz: ±0
	\item Zugang zu Sonderfertigkeiten wie bisher
\end{itemize}
Die Vorteile \enquote{Meisterhandwerk} und \enquote{Schutzgeist} bleiben unangetastet.

Der Vorteil \enquote{übernatürliche Begabung} schaltet keine Standardzauber nach Liber Cantiones mehr frei, sondern auf den Charakter und dessen Hintergrund maßgeschneiderte \enquote{Minizauber}, deren Wirkungen und sonstige Eigenschaften wie Zauberdauer und Kosten sich an den im Liber Cantiones beschriebenen orientieren können, aber nicht müssen. Der Vorteil kann nur einmal erworben werden, kostet \SI{2}{\GP} und gestattet die Generierung von bis zu 5 Zaubern, auf die \SI{5}{\ZfP} frei verteilt werden können. Der Vorteil stellt die Repräsentation \enquote{Magiedilettant (Name)} zur Verfügung.

\subsection{Repräsentation \enquote{Magiedilettant (Name)} (Dil)}
Die Komponenten der magiedilettantischen Repräsentation sind Konzentration und Sicht. Es gibt keine zentrale Komponente. In der magiedilettantischen Repräsentation sind nur die spontanen Modifikationen \enquote{Reichweite} und \enquote{Erzwingen} möglich. Die Leiteigenschaft für Magiedilettanten ist Intuition. Magiedilettanten sind in der Lage eine bestimmte Menge fehlender Zauberfertigkeitspunkte zum Bestehen einer Zauberprobe durch den Einsatz weiterer Astralpunkte auszugleichen. Ein ausgeglichener Zauber gilt immer als \enquote{gerade so geschafft}, was bedeutet, dass immer genau \SI{0}{\ZfPstern} und nicht mehr übrig behalten werden. \enquote{Ausgleichen} zählt nicht als spontane Modifikation. Fehlgeschlagene Zauber kosten ¾ der vollen Spruchkosten. Zauber in magiedilettantischer Form bestehen immer nur aus ihrer Grundwirkung, es gibt keine Varianten. Varianten sind regeltechnisch eigenständige Zauber, die getrennt voneinander gesteigert werden müssen. Zauber können nicht schriftlich fixiert werden. Es können keine Zauber anderer Repräsentationen erlernt werden, darunter fallen explizit auch Zauber anderer Magiedilettanten (der Zusatz \enquote{(Name)} soll dies deutlich machen). Es ist ebenfalls nicht möglich, dass ein Magiedilettant mit gleichen Zaubern und höherem ZfW als Lehrmeister für einen anderen Magiedilettanten fungiert. Steigerung ist ausschließlich im Selbststudium möglich; ob Halbelfen ihre Sprüche im Salasandra steigern können, ist Meisterentscheid. Zauber mit den Merkmalen \enquote{Beschwörung}\footnote{Ausnahme: Weiße Mähn' und Gold'ner Huf.}, \enquote{Herrschaft} oder \enquote{Limbus} sind in magiedilettantischer Repräsentation nicht möglich. Zauber mit einer Komplexität von E oder höher sind in magiedilettantischer Repräsentation nicht möglich.

\subsection{Dilettantenzauber}
Grundsätzlich trifft die Liste der für Viertelzauberer zur Verfügung stehenden Zauber (WdH 258) weiterhin zu, die Sprüche sind aber in ihrer Wirkung eingeschränkt: während der Balsam Salabunde neben der Grundwirkung noch zwei Varianten kennt, stehen beide Varianten dem Viertelzauberer nicht zur Verfügung. Je nach dem wie viele Varianten eines Spruches wegfallen, kann sich dies nach Meisterentscheid auf die Komplexität auswirken. Grundsätzlich gilt für Viertelzauberer aber, dass nach Rücksprache mit dem Meister auch vollkommen neue Zauber kreiert werden können, die kein Pendant im Liber Cantiones haben.

Für die Wahl einer Variante beziehungsweise die Erstellung eines Zaubers empfehle ich die folgenden Richtlinien:
\begin{itemize}
	\item Varianten mit einer Erschwernis von bis etwa +7 ohne Mindest-ZfW dürfen gewählt werden. Die Erschwernis der Variante entfällt in der magiedilettantischen Version des Spruches.
	\item Varianten mit höherer Erschwernis oder Mindest-ZfWs von 4 oder höher sollten Viertelzauberern nicht zur Verfügung stehen.
	\item Reversalierte Formen einer Wirkung sind gestattet. Reversalierte Formen sind gegenüber der Grundform nicht erschwert; auch hier entfällt die Erschwernis für Viertelzauberer.
	\item Varianten müssen in allen Repräsentationen vorliegen. Ausnahme sind hier elfische Viertelzauberer, die zusätzlich all jene Varianten wählen können, die in der elfischen Repräsentation vorkommen.
	\item Die Reichweite ist grundsätzlich auf \enquote{selbst} begrenzt. Ausnahmen sind Zauber, die überwiegend oder ausschließlich nicht auf sich selbst gewirkt werden wie zum Beispiel der \enquote{Blitz dich find} oder der \enquote{Odem Arcanum}.
	\item Abweichend zur offiziellen Liste der für Viertelzauberer gestatteten Zauber (WdH 258) sind nun auch folgende Zauber oder Varianten möglich:
\begin{itemize}
	\item Aeolitus (nur Halbelfen)
	\item Arachnea, Variante \enquote{Schmetterlingssammler} mit der zusätzlichen Einschränkung, dass der Zauber auf eine bestimmte Tierart festgelegt ist
	\item Blick in die Vergangenheit
	\item Eins mit der Natur
	\item Elementarbann, Variante \enquote{Mindergeister bannen}
	\item Geisterruf
	\item Haselbusch und Ginsterkraut, Variante \enquote{Schnelle Schlinge} (nur Halbelfen)
	\item Hilfreiche Tatze, rettende Schwinge (nur Halbelfen)
	\item Manifesto mit der zusätzlichen Einschränkung, dass der Zauber auf ein bestimmtes Element festgelegt ist
	\item Somnigravis (nicht mehr exklusiv für Halbelfen)
	\item Tempus Stasis mit der Einschränkung auf ein Zielobjekt oder -wesen
	\item Unberührt von Satinav
	\item Weiße Mähn' und Gold'ner Huf (nur Halbelfen)
	\item Zauberwesen der Natur
\end{itemize}

\end{itemize}
Der Kerngedanke hinter diesen Regeln ist, dass jeder Viertelzauberer zwar magische Wirkungen hervorrufen kann, aber diese nur zufällig mit bekannten Sprüchen übereinstimmen sollten.

\subsection{Ausgleichen misslungener Zauberproben}
Ein Viertelzauberer kann Astralpunkte investieren, um eine misslungene Zauberprobe mit genau \SI{0}{\ZfPstern} zu bestehen. Dies gilt nur dann, wenn die Probe mit mindestens \SI{0}{\ZfP} gewürfelt wird. Das bedeutet, dass Proben, bei denen zum Beispiel durch Magieresistenz oder Antimagie \SI{-2}{\ZfP} zur Verfügung stehen, nicht ausgeglichen werden können. Die Erschwernis von +2 auf jeden einzelnen Wurf bleibt daher in vollem Umfange aktiv und kann nicht ausgeglichen werden. Die fehlenden Zauberfertigkeitspunkte werden 1:1 mit Astralpunkten bezahlt zuzüglich 1W3~−~\SI{1}{\AsP}. Ausgeglichene Zauber behalten immer genau \SI{0}{\ZfPstern} übrig. Der Held entscheidet in der Regel selbst, ob er versuchen möchte die misslungene Probe auszugleichen. Scheitert dies an fehlenden Astralpunkten, so scheitert der Zauber mit den üblichen Folgen. Nach Meisterentscheid aktiviert sich dieser Ausgleichsmechanismus automatisch, zum Beispiel in emotional aufgeladenen Situationen.

\subsection{Beispiele}
Die noch in Ausbildung befindliche Mathematica Oleana hat ein Geheimrezept, wenn es darum geht die Nacht vor einer wichtigen Prüfung gut zu schlafen: vor dem Schlafengehen setzt sie sich auf ihr Bett und stellt sich vor, dass sie sofort einschlafen kann und am nächsten Morgen nach einer erholsamen Nacht aufwacht. Kurz darauf merkt sie auch schon, dass sie sehr müde wird, und noch während sie so bei sich zufrieden denkt: \enquote{Es hat funktioniert...}, kuschelt sie sich noch schnell in ihre Decke und unterliegt vollkommen der Wirkung des \enquote{Ruhe Körper, Ruhe Geist}, den sie auf sich selbst gewirkt hatte. Regeltechnisch hat sie dabei eine unmodifizierte Probe auf 13/10/11 mit einem ZfW von 4 bestanden. Mit einem Würfelergebnis von 8/13/13 hat sie die Probe durch Investition von 1~+~1W3~−~1 Astralpunkten mit \SI{0}{\ZfPstern} geschafft.

Die Dorfälteste Delusia wird von den Bewohnern ihres Dorfes, aber auch umliegender Dörfer und Städte, immer dann zu Rate gezogen, wenn es zu spuken scheint. Meistens ist es nichts, aber hin und wieder existiert doch noch ein Geist in der Dritten Sphäre, den Delusia dann mit Hilfe eines \enquote{Geisterruf} rufen kann, um so unter Umständen zu erfahren, wieso dieser Geist in dieser Sphäre geblieben ist. Regeltechnisch müssen Delusia dafür unmodifizierte Proben auf 15/15/13 mit einem ZfW von 11 gelingen. Damit ist sie in der Lage die Anrufungsschwierigkeit aller beschriebenen Geister (WdZ 204/205) auszugleichen und könnte Würfelpech immer noch mit weiteren Astralpunkten ausgleichen.

Der junge Welf hat schon als Kind erkannt, dass sich seine Eltern sehr um seine Anfälligkeit für Verletzungen sorgen und das auch die Familienkasse belastet. Nachdem er eines Tages beim Spielen mit anderen Kindern gestürzt war, hielt er sich für einige Minuten die blutende Wunde und wünschte sich, dass er seinen Eltern nicht zur Last fallen soll. Er bemerkte, dass die Wunde sich alsbald zu schließen begann und freute sich, weil er sich dachte er wäre mit einer so schnellen Wundheilung endlich so normal wie die anderen Kinder auch. Dass er einen ausgeglichenen (\SI{-2}{\ZfPstern} durch 2~+~1W3~−~\SI{1}{\AsP}) unmodifizierten \enquote{Balsam Salabunde} auf sich selbst gewirkt hatte, war ihm nicht klar.

Seine Geschwister haben sich einen Spaß daraus gemacht den kleinen Miljan häufig zu erschrecken. Auch wenn er dadurch im Erwachsenenalter ein wenig sonderbar~--~mancher würde gar sagen er wäre paranoid~--~geworden ist, hat sich dies in einer besonderen Begabung niedergeschlagen. Nicht nur verfügt er über einen ausgezeichneten Gefahreninstinkt, da die kleinen Späße das ein oder andere Mal auch deutlich zu weit gegangen sind, sondern auch über die Fähigkeit quasi im Affekt Personen und Gegenstände für wenige Augenblicke \enquote{einzufrieren}. Regeltechnisch verbirgt sich hinter dieser äußerst interessanten Fähigkeit eine übernatürliche Begabung für eine auf Zielobjekte und -wesen wirkende Variante des \enquote{Tempus Stasis}. Eine willentliche Auslösung sollte nicht von Spielbeginn an zur Verfügung stehen und sich durch einen rollenspielerischen Bewusstwerdungsprozess und entsprechende Steigerungen der Begabung auszeichnen. Für die Ausgestaltung dieser Affektzauberei bieten sich Anleihen zu den Expertenregeln zur Elfenmagie an (WdZ 320).
