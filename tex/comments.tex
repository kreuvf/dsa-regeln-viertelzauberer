\subsection{Kommentar}
Ich habe versucht eine große Kompatibilität zu den offiziellen Regeln zu erhalten, die Regeln für Viertelzauberer aber feiner auszuarbeiten und damit offene Fragen auszuräumen. Das Ziel ist es insgesamt jeden Viertelzauberer zu einem Individualisten zu machen, der seine Fähigkeiten in der Regel sicher einsetzen kann, auch wenn er eben nicht genau weiß wie das funktioniert -- ähnlich wie viele von uns nicht wissen wie ihr Verdauungstrakt funktioniert.

Für die Anwendbarkeit dieser Regeln auf ausgebildete Viertelzauberer sehe ich keine Probleme. Die GP-Kosten könnten leicht abweichen, was aber durch eine entsprechende Verrechnung \SI{1}{\GP}&nbsp;=&nbsp;\SI{50}{\AP} behoben werden kann. Die AP werden dann einfach zu den Start-AP hinzuaddiert.

\paragraph{Vorteil \enquote{Viertelzauberer}}
Die schwammige Formulierung zur Kostenberechnung des Vorteils \enquote{Viertelzauberer} (WdH 257/258) habe ich durch klarere Erklärungen zu den nötigen Randbedingungen ersetzt. Die Bewusstwerdung kann weiterhin als rollenspielerisches Element genutzt werden. Dass eine übernatürliche Begabung zwangsweise die teurere Variante nach sich zieht, erkläre ich damit, dass übernatürlich begabte Charaktere sich ihrer Begabung immer bewusst sind, wenn auch die Natur dieser Begabung in der Regel unerkannt bleibt: sie wissen, dass sie bestimmte Handlungen durchführen können, wissen aber nicht unbedingt, dass Magie im Spiel ist. Dass dem so ist, erkläre ich damit, dass nach den offiziellen Regeln eine übernatürliche Begabung auf einem Wert von 3 startet (WdZ 257), was bedeutet, dass der Charakter eine gewisse Zeit in diese Fähigkeit investiert haben muss. Er ahnt vielleicht, dass diese Fähigkeit magischen Ursprungs ist, weiß es aber nicht. Sollte er auf eine Gruppe mit einem Magier treffen, kann dieser sicher leicht herausfinden, ob eine magische Begabung vorliegt oder nicht.

Um Viertelzauberer etwas mächtiger zu machen, habe ich die Startastralenergie um drei Punkte angehoben, sodass ein durchschnittlicher Held (12 aus MU, IN, CH) mit \SI{15}{\AsP} startet, was etwa vier Ruhephasen braucht, um sich vollständig zu regenerieren. Umgerechnet in Zauber bedeutet dies, dass er sich (ohne Ausgleichen) jeden zweiten Tag eine Wunde mit einem Balsam heilen oder einen Angreifer mittels Horriphobus einschüchtern kann.

\paragraph{Vorteil \enquote{Übernatürliche Begabung}}
Da Viertelzauberer sowas wie Einzelstücke sind, passt es nicht in mein Aventurienbild, wenn sich der Viertelzauberer einfach plump Zauber nach Liber Cantiones heraussuchen darf. Zumal wie ja oben angesprochen mitunter stark unterschiedliche Varianten in einen Zauber gepackt wurden und ich mir so beim besten Willen nicht vorstellen kann, dass ein Viertelzauberer von sich aus auf all diese Varianten stößt, wenn er einen höheren ZfW erreicht hat. Die Zauber, die ein Viertelzauberer wirken kann, sollten immer mit dem intuitiv umgesetzten Wunsch nach einer Beeinflussung der Welt zu tun haben. Das heißt auch, dass die magische Wirkung tief mit dem Wirker verbunden ist, was Vor- und Nachteile bedeutet. In gewisser Weise ähnelt das der elfischen Magie, wenngleich die harmonische Komponente fehlt. Bei der Wahl der möglichen Sprüche gestatten die vorgestellten Regeln daher auch größere Freiheiten, letztlich bleibt es aber immer Meisterentscheid, ob ein bestimmter Zauber oder eine bestimmte Wirkung erlaubt ist. Daher sind auch die von mir zusätzlich erlaubten Sprüche nicht in Stein gemeißelt. Hier ist natürlich auch die Kreativität der Spieler gefragt, um den Meister davon zu überzeugen, dass ein bestimmter Spruch oder eine bestimmte Wirkung doch sehr gut zum Charakter passt.

Des Weiteren habe ich die Kosten für übernatürliche Begabungen geändert. Zum einen können jetzt bis zu fünf Sprüche für \SI{2}{\GP} gewählt werden, zum anderen ist die Steigerung deutlich günstiger als vorher: als Extremfall sei auf die Steigerung des Flim Flam von 10 auf 11 verwiesen, die nun nicht mehr \SI{280}{\AP}, sondern nur noch \SI{43}{\AP} kostet. Dabei sollte man immer im Hinterkopf haben, dass ein Viertelzauberer weniger von einer Steigerung hat als Halb- oder Vollzauberer, da Regeneration und Astralenergie für Viertelzauberer zeitlebens vergleichsweise gering ausfallen werden und eine Steigerung zwar zu stärkeren Effekten führen kann, aber nicht zu häufigerem Einsatz. Sollte ein Spieler einen Charakter erschaffen wollen, der häufiger Magie wirken kann, ist ein Viertelzauberer meiner Meinung nach der falsche Weg; stattdessen sollte dieser Spieler sich eher für Halb- oder sogar Vollzauberer entscheiden.

Alternativ zu den normalen Steigerungsregeln wäre es auch denkbar übernatürliche Begabungen über eine Art \textit{learning by doing} zu steigern. Dabei steigt der ZfW einer übernatürlichen Begabung immer dann, wenn ein Lernschwellenwert überschritten wird. Jedes Mal, wenn die übernatürliche Begabung eingesetzt wird und ihr Einsatz gelingt, kann der Spieler eine Intuitionsprobe würfeln. Diese Probe wird erschwert um die <abbr title="aus der Zauberprobe übrig behaltener Zauberfertigkeitspunkt">ZfP*</abbr> aus der Zauberprobe. Gelingt die Intuitionsprobe, erhält der Charakter die übrigen Punkte aus der Intuitionsprobe auf das Steigerungskonto der übernatürlichen Begabung (mindestens einen Punkt). Die Idee dahinter ist, dass der Charakter besonders dann etwas über seine Begabung lernen kann, wenn er an die Grenze seiner Fähigkeiten stößt, was sich sogar mit irdischen Erkenntnissen aus der Lernpsychologie deckt. Eine entsprechende Steigerungskostentabelle zu erstellen, ist mir aber für diesen Artikel zu aufwendig -- sowas würde sich aber exzellent für ein \enquote{AP-freies Spiel} eignen, eine Idee, die ich in Zukunft gern weiter ausarbeiten würde und für die ich auch schon einige Ideen im Hinterkopf habe.

\paragraph{Repräsentation \enquote{Magiedilettant (Name)}}
Ein Viertelzauberer sollte seine Fähigkeiten bei Abwesenheit externer Erschwernisse (Magieresistenz, Antimagie, etc.) routiniert einsetzen können, weshalb ich Viertelzauberern gestatte misslungene Proben unter bestimmten Umständen auszugleichen. Durch die meist schlechte Regeneration von Astralpunkten begrenzt sich der Viertelzauberer damit selbst und wird dadurch nicht so schnell wieder seine Fähigkeiten einsetzen können wie ein Halb- oder Vollzauberer. Und selbst dann können etwas teurere Zauber immer noch danebengehen. Der Ausgleichsmechanismus soll darüber hinaus übernatürliche Begabungen näher an die Effekte eines Meisterhandwerks oder des Schutzgeistes heranbringen.

Was auch direkt zum neuen Nachteil der Viertelzauberer, genauer gesagt: der Repräsentation Magiedilettant, überleitet: Viertelzauberer haben nur geringe Kontrolle über ihren astralen Fluss, wenn sie erst einmal angefangen haben sich die Wirkung herbeizuwünschen. Dadurch verlieren sie für fehlgeschlagene Zauber nicht wie üblich die Hälfte der eigentlich fälligen Astralpunkte, sondern drei Viertel. Sie konzentrieren sich auch dann noch auf die gewünschte Wirkung, wenn ein Halb- oder Vollzauberer schon längst gemerkt hat, dass die Kraftfäden falsch gesponnen wurden und sich so nichts machen lässt.

Ein Teil der Regeln zur neu erschaffenen (Pseudo-)Repräsentation \enquote{Magiedilettant (Name)} sind direkt aus den offiziellen Regeln entnommen. Dadurch, dass Zauber eines Viertelzauberers jetzt in der eigenen Repräsentation vorliegen, gilt automatisch, dass eine Steigerung auf \enquote{höchste an der Probe beteiligte Eigenschaft} +3 erlaubt ist, was für gewöhnlich eine Obergrenze im Bereich von 21 bis 24 bedeutet. Die Merkmalseinschränkungen habe ich aus den Merkmalen der erlaubten Zauber abgeleitet. Die drei Hauptmerkmale der offiziell zulässigen Zauber sind \enquote{Eigenschaften} (36\%), \enquote{Einfluss} (24\%) und \enquote{Hellsicht} (19\%) und die drei nur einmal verwendeten Merkmale sind \enquote{Antimagie} \enquote{Metamagie} und \enquote{Schaden}. Die Prozentangaben geben den Anteil an Zaubern mit diesem Merkmal an allen wählbaren Zaubern an; dadurch bedingt ist die Summe höher als 100\%. Die Komplexitätsgrenze ist ebenfalls aus den offiziell zulässigen Zaubern übernommen. Besonders wichtig ist mir, dass Viertelzauberer auf sich allein gestellt sind, wenn es darum geht ihre Zauber zu steigern. Dadurch ist die Steigerung ihrer Sprüche auf Spalte F beschränkt: Komplexität des Spruches D + zwei Spalten nach rechts für Steigerung auf 11 oder höher. Da es sicher einige ausgefuchste Spieler geben wird, die argumentieren werden, dass \enquote{Magiedilettant (Alrik)} für alle magiedilettantischen Alriks gilt, nochmal explizit: auch gleiche Namen gestatten es nicht die Randbedingungen der Repräsentation auszuhebeln.

Durch die Personalisierung der Repräsentation steht des Weiteren auch eine im sonstigen regeltechnischen Rahmen stimmige Erklärung für die Art und Weise wie Magiedilettanten Zauber wirken. Bei der Analyse eines magiedilettantischen Zaubers mittels \enquote{Analys} könnte der Meister von Würfeln abhängig machen, welcher bekannten Repräsentation der Zauber am ähnlichsten sieht. Folgende zwei Möglichkeiten schlage ich vor: zum einen kann der Meister von einem W20 abhängig machen wie ähnlich die Repräsentation eines bestimmten Magiedilettanten mit einer dem Hellsichtmagier bekannten ist und die üblichen Zuschläge (WdZ 52) dafür verhängen. Alternativ zu einem W20 kann anhand von nachfolgender Tabelle auch mit einem W6 entschieden werden, welcher Repräsentation ein bestimmter Zauber am ähnlichsten ist. Dies sollte für zukünftige Fragen notiert werden.
\begin{table}
	\centering
	\caption[Kurzbeschreibung für Verzeichnis]{Langbezeichnung für direkte Anzeige\label{identifier-vergeben}}
	\begin{tabular}{SPALTENDEFINIEREN!}
		\toprule

<th>Augenzahl</th>
<th colspan="2">Repräsentation</th>


<td>1</td>
<td colspan="2">Gildenmagische Repräsentation</td>


<td>2</td>
<td colspan="2">Scharlatanische Repräsentation</td>


<td>3</td>
<td colspan="2">Satuarische Repräsentation</td>


<td>4</td>
<td colspan="2">Druidische Repräsentation</td>


<td>5</td>
<td>Geodische Repräsentation</td>
<td rowspan="2">Halbelfen: elfische Repräsentation</td>


<td>6</td>
<td>Elfische Repräsentation</td>

		\bottomrule
	\end{tabular}
\end{table}

Nachdem ich den Abschnitt zur Repräsentation \enquote{Magiedilettant} geschrieben hatte, fragte ich mich wie Zauber aus übernatürlichen Begabungen wohl in der \href{http://www.helden-software.de/}{Helden-Software} implementiert sind und sah, dass Magiedilettantenzauber der Repräsentation \enquote{Dil} angehören. Es ist schön zu sehen, dass ich offenbar nicht der Erste bin, der sich über die Repräsentationslosigkeit magiedilettantischer Zauber gewundert hat.

\paragraph{Vorteil \enquote{Meisterhandwerk}}
Besonders geärgert hat es mich im WdH zu entdecken, dass Gjalskerländer Tierkrieger über ein Meisterhandwerk in einer Eigenschaft verfügen und all das ohne auch nur den Hauch einer Erklärung wie sowas regeltechnisch ablaufen soll. Da es bereits einen ähnlichen Vorteil, die Gabe Kräfteschub für \SI{10}{\GP}, gibt, will ich mit den Regeln für ein Meisterhandwerk einer Eigenschaft dafür sorgen, dass Kosten und Nutzen in einem angemessenen Verhältnis stehen. Dadurch, dass Meisterhandwerke für Halb- und Vollzauberer teurer sind, ist diese Ausprägung eines Meisterhandwerks rein kostenmäßig wahrscheinlich sehr unattraktiv. Aufgrund der aber eher begrenzten Wirkung passt ein solches Meisterhandwerk immerhin sehr gut zu einem Viertelzauberer.

Da gewisse Nachteile auch auf Eigenschaften durchschlagen, sollte der Meister entscheiden, ob ein bestimmter Nachteil einem Meisterhandwerk in einer Eigenschaft im Wege steht oder nicht. Im Falle der Eigenschaft GS könnte der Meister dem Charakter also verbieten behäbig, einbeinig, fettleibig, kleinwüchsig oder lahm zu sein. Es wäre aber auch gut denkbar, dass es zum Hintergrund des kleinwüchsigen, fetten oder einbeinigen Charakters passt, dass dieser gelernt hat seinen Geschwindigkeitsnachteil durch das Meisterhandwerk auszugleichen. Vielleicht ist es sogar gerade eine herausstechende Eigenschaft des Charakters so auszusehen als könnte man vor ihm einfach weglaufen, wenn dem aber tatsächlich nicht so ist.
