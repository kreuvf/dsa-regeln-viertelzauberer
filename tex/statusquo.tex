\section{Status quo: Viertelzauberer in DSA4.1}
Als Viertelzauberer oder Magiedilettanten werden magisch begabte Wesen verstanden, deren magisches Talent unentdeckt blieb, nicht ausgebildet wurde oder nicht stark genug für eine Ausbildung war (WdZ 35).

Ein Viertelzauberer kann über magische Meisterhandwerke, einen Schutzgeist und übernatürliche Begabungen verfügen. Schutzgeist und Meisterhandwerk sind ausreichend gut geregelt, aber übernatürliche Begabungen aber bieten mehr als genug Anlass zur Kritik. Einige dieser Punkte werden bereits in der \href{http://www.wiki-aventurica.de/wiki/Hausregelsammlung/NADSA/Viertelzauberer}{NADSA-Hausregelsammlung auf Wiki Aventurica} angesprochen.

\subsection{Kosten}
Übernatürliche Begabungen werden während der Heldenerschaffung nach Spalte F gesteigert (WdH 257). Spalte F ist sonst nur den kompliziertesten Zaubern in eigener Repräsentation vorbehalten (WdS 169). Da Viertelzauberer keine eigene Repräsentation besitzen und sich ihre magischen Wirkungen intuitiv entfaltet haben, ist eine Steigerung stimmig nur nach den Regeln des Selbststudiums möglich (WdS 167, WdS 170): Steigerung bis zu einem ZfW von 10 ist um eine Spalte (= Spalte G) erschwert, Steigerungen jenseits von 10 um zwei Spalten (= Spalte H). Daraus ergibt sich die Zeit zum Steigern eines Zaubers. Sie beträgt halb so viele Zeiteinheiten wie Abenteuerpunkte benötigt werden (WdS 170). Da eine $\si{ZE} = \SI{2}{\h}$, ist die Umrechnung einfach: \SI{1}{\AP} entspricht \SI{1}{\h}. Allein das Steigern von 3 auf 4 würde daher für einen Viertelzauberer \SI{42}{\h} beanspruchen, was eine Woche intensiver Auseinandersetzung bedeutet. Bei der Steigerung von 11 auf 12 wären es bereits \SI{320}{\h}, was acht Wochen intensiver Auseinandersetzung bedeutet.

Generierungskosten für übernatürlich begabte Viertelzauberer sind verglichen mit Halb- und Vollzauberern sehr hoch. Tabelle~\ref{zauberer-gp} enthält eine Übersicht über die Generierungskosten.
\begin{table}[b]
	\centering
	\caption[Generierungskosten und ausgewählte Auswirkungen der Vorteile Viertel-, Halb- und Vollzauberer]{Gegenüberstellung der Generierungskosten für Viertel-, Halb- und Vollzauberer und die mit den jeweiligen Vorteilen verbundenen Auswirkungen auf die Astralenergie, die Magieresistenz und die Zahl der vorhandenen Hauszauber.\label{zauberer-gp}}
	\begin{tabular}{lllll}
		\toprule
		Vorteil & Kosten & Astralenergie & Magieresistenz & Hauszauber \\
		\hline
		Viertelzauberer & \SI{5}{\GP} & Grundmenge \SI{-6}{\AsP} & ±0 & 0 \\
		Halbzauberer & \SI{10}{\GP} & Grundmenge +\SI{6}{\AsP} & +1 & 5 \\
		Vollzauberer & \SI{20}{\GP} & Grundmenge +\SI{12}{\AsP} & +2 & 7 \\
		\bottomrule
	\end{tabular}
\end{table}
Wichtig hierbei ist, dass in dieser Tabelle noch nicht eingerechnet ist, dass Viertelzauberer ihre maximal fünf Sprüche für je \SI{1}{\GP} erwerben müssen, was die Professionskosten auf mindestens \SI{6}{\GP} erhöht. In der Gesamtbetrachtung müssten noch dazu einige weitere Punkte eingehen:
\begin{itemize}
	\item Randbedingungen der Repräsentation
	\item zugängliche magische Sonderfertigkeiten
	\item zusätzliche Zahl weiterer aktivierbarer Zauber
	\item Ritualkenntnis
	\item Steigerungskosten der aktivierten Zauber
\end{itemize}
Nur im ersten Punkt, den Randbedingungen der Repräsentation, kann der Viertelzauberer marginal punkten: keine Geste, keine Formel, höchstens Sicht und Konzentration werden benötigt. In allen anderen Punkten, auch was die Repräsentation anbelangt, liegen Viertelzauberer weit hinter anderen Magiebegabten zurück.

\subsection{Drei Arten von Viertelzauberern}
Aus den Beschreibungen in den offiziellen Publikationen lassen sich drei Arten von Viertelzauberern ableiten (WdH 258):
\begin{itemize}
	\item Ausgebildete Viertelzauberer: Alchimisten, Derwische, Ferkina-Besessene, Gjalskerländer Tierkrieger, Zaubertänzer
	\item Viertelzauberer (bewusst)
	\item Viertelzauberer (unterbewusst)
\end{itemize}
Die Beschreibung der hier als \enquote{unterbewusst} bezeichneten Kategorie ist leider sehr schwammig gehalten: der Vorteil \enquote{Viertelzauberer} kann statt für \SI{5}{\GP} auch für \SI{3}{\GP} erworben werden. Die Bedingung dafür ist, dass der Viertelzauberer nichts von seiner Begabung weiß und der Meister entscheiden kann, wann er etwas davon erfährt (WdH 258). Die genaue spiel- und regeltechnische Bedeutung dessen wird leider nicht weiter ausgeführt. Kann der Viertelzauberer seine Fähigkeiten nicht einsetzen? Kann er sie einsetzen, aber nicht steigern, solange er sich dessen nicht bewusst ist? Kann der Meister dem Spieler vorschreiben, wann die Fähigkeiten eingesetzt werden müssen, da der Charakter dies nicht aktiv steuern kann? Oder funktionieren unterbewusste Viertelzauberer nur mit Schutzgeist und Meisterhandwerk?

\subsection{Zauber- oder Wirkungsauswahl}
Obwohl Viertelzauberer keine eigene Repräsentation besitzen, unausgebildete Viertelzauberer noch dazu keinerlei magietheoretische Kenntnisse aufweisen und nicht in der Lage sind weitere Zaubersprüche zu erlernen, sind die möglichen Zauberwirkungen auf einen Kreis von 67 Zaubern, davon 8 exklusiv für Halbelfen, beschränkt (WdH 258). Vor dem Hintergrund, dass die Zauberwirkung intuitiv erlernt wird, ist die harte Beschränkung auf einen definierten Kreis von Zaubern unsinnig. Auf der anderen Seite stehen Viertelzauberern sämtliche Varianten eines Zaubers, die in sämtlichen Repräsentationen vorhanden sind, offen (WdZ 38). Zudem sind nur die spontanen Modifikationen \enquote{Reichweite} und \enquote{Erzwingen} möglich (WdZ 38). Dies führt zu der absurden Situation, dass ein Viertelzauberer ebenfalls sämtliche Varianten des Motoricus beherrscht, der so unterschiedliche Auswirkungen wie \enquote{Fesselfeld} und \enquote{Rote und Weiße Kamele} enthält.
