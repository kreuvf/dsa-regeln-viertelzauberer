\section{Danksagung}
Ich möchte mich bei meinem Meister, Matthias Petzold, für seine Anmerkungen während der Entstehung des Artikels danken.

\section{Hinweise}
Dieser Artikel ist entstanden, weil ich \enquote{mal eben schnell} ein paar Ungereimtheiten beim Viertelzauberer beheben wollte. Je mehr ich aber nachgedacht und geschrieben habe, desto mehr Punkte sind mir aufgefallen. Natürlich kann ich aber auch etwas übersehen oder eine offizielle Regel falsch verstanden haben.

Falls es also Lob, Kritik oder Anmerkungen zu diesem Artikel gibt, freue ich mich jederzeit über eine \href{http://kreuvf.de/impressum.php}{E-Mail}.

\subsection{<abbr title="Portable Document Format">PDF</abbr>}
Eine <a href="" title="PDF-Version von \enquote{DSA: Regeln für Viertelzauberer}">PDF-Version dieses Artikels} ist ebenfalls verfügbar. Die <abbr title="Portable Document Format">PDF</abbr> wurde mit \href{https://de.wikipedia.org/wiki/TeX_Live}{TeX Live} 2014, \href{https://de.wikipedia.org/wiki/XeTeX}{XeTeX} und unter Nutzung der Schriftfamilie &bdquo;\href{https://de.wikipedia.org/wiki/Linux_Libertine}{Linux Libertine}&ldquo; unter \href{https://de.wikipedia.org/wiki/Xubuntu}{Xubuntu} erstellt.

\subsection{Version}
Dieses Dokument wird mit dem Versionsverwaltungsprogramm \href{http://git-scm.com/}{Git} verwaltet. Die Versionen dieses Dokuments können in einem \href{https://github.com/kreuvf/dsa-regeln-viertelzauberer}{Klon des Repositorys} gefunden werden. Da mein Veröffentlichungsprozess noch nicht gut genug automatisiert ist, findet sich in der Webversion dieses Artikels keine Versionsangabe, in der PDF-Version hingegen schon.
