\section{Danksagung}
Ich möchte mich bei meinem Meister, Matthias Petzold, für seine Anmerkungen während der Entstehung des Artikels danken.

\section{Hinweise}
Dieser Artikel ist entstanden, weil ich \enquote{mal eben schnell} ein paar Ungereimtheiten beim Viertelzauberer beheben wollte. Je mehr ich aber nachgedacht und geschrieben habe, desto mehr Punkte sind mir aufgefallen. Natürlich kann ich aber auch etwas übersehen oder eine offizielle Regel falsch verstanden haben.

Falls es also Lob, Kritik oder Anmerkungen zu diesem Artikel gibt, freue ich mich jederzeit über eine \href{https://kreuvf.de/impressum.php}{E-Mail}.

\paragraph{Lizenz}
Dieser Artikel ist unter der \href{https://creativecommons.org/licenses/by-sa/3.0/de/}{CC-BY-SA-3.0-Lizenz} veröffentlicht. Es folgt eine rechtlich nicht bindende Zusammenfassung in allgemein verständlicher Sprache:
\begin{itemize}
	\item Der Artikel darf in jedem Format oder Medium vervielfältigt und weiterverbreitet werden.
	\item Der Artikel darf verändert oder als Grundlage für eigene Werke genutzt werden. Der Zweck ist dabei nicht von Belang. Auch eine kommerzielle Verwertung ist erlaubt.
	\item Sie müssen angemessene Urheber- und Rechteangaben machen, einen Link auf die Lizenz anfügen und deutlich machen, ob Veränderungen vorgenommen wurden. Diese Angaben dürfen in jeder angemessenen Art und Weise gemacht werden, allerdings nicht so, dass der Eindruck entsteht, der Lizenzgeber unterstütze gerade Sie oder Ihre Nutzung besonders.
	\item Verändern Sie diesen Artikel oder benutzen Sie ihn als Grundlage für eigene Werke, so dürfen Sie den veränderten Artikel oder Ihr neues Werk nur unter denselben Bedingungen weitergeben, unter die Sie diesen Artikel nutzen dürfen.
	\item Sie dürfen keine zusätzlichen Klauseln oder technische Verfahren einsetzen, die anderen rechtlich irgendetwas untersagen, was die Lizenz erlaubt.
\end{itemize}

\subsection{Webversion}
Eine \href{https://blog.kreuvf.de/2015/06/04/dsa-regeln-fur-viertelzauberer/}{Web-Version dieses Artikels} ist ebenfalls verfügbar.

\subsection{Stand}
Dieses Dokument wird mit dem Versionsverwaltungsprogramm \href{https://git-scm.com/}{Git} verwaltet. Die Versionen dieses Dokuments können in einem \href{https://github.com/kreuvf/dsa-regeln-viertelzauberer}{Klon des Repositorys} gefunden werden. Die Version dieses Dokuments lautet \enquote{\gitAbbrevHash{}}\footnote{Die vollständige Versionsnummer lautet: \gitHash{}.} vom \gitAuthorIsoDate{} von \gitAuthorName{} (\href{mailto:webmaster@kreuvf.de}{webmaster@kreuvf.de}), der verwendete Branch ist \gitBranch.